\section{SRA}
\subsection{Die Vorgabe}
\begin{frame} %%Eine Folie
  \frametitle{Aufgabe} %%Folientitel

	\highlighton{Leitung des Teams Content \& Layout}
%   Auflistung mit Punkt davor
  	\begin{itemize}
   		\item Verteilung von Aufgaben
   		\item Bildung von Unterteams
   		\item Fertigstellung und Abgabe des SRA-Dokuments
   		\item Kontrolle und Bewertung der Ergebnisse
  	\end{itemize}
	\bigskip \highlighton{Mitarbeit am SRA}
	\begin{itemize}
		\item Punkt F4: Lehrer/Mentor mit Christian Dauerer
		\item Punkt S: Systemanforderungen
	\end{itemize}
\end{frame}

\subsection{Die Formulierung}
\begin{frame} %%Eine Folie
  \frametitle{Gruppeneinteilung} %%Folientitel
  
  Folgende Gruppeneinteilung wurden im Team Content \& Layout vorgenommen:
% Definitionsblock
  \begin{block}{}
  	\begin{enumerate}
  		\item Matthias Götz, Sven Liebl
  		\item Matthias Birnthaler, Tobias Schwindl
  		\item Nguyen Hoan, Stefan Holz
  		\item Christian Dauerer, Alexander Strobl
  	\end{enumerate}
  \end{block}
  Die Aufteilung in 2er Teams soll das Teamwork verbessern und durch die gegenseitige Kontrolle der Teammitglieder die Ergebnisse verbessern und gleichzeitig weniger Fehler hervorrufen.
\end{frame}

\begin{frame} %%Eine Folie
  \frametitle{Text für die SRA} %%Folientitel
  
  Zusammen mit Christian Dauerer wurden folgende Texte formuliert und in die SRA eingebaut:
% Definitionsblock
  \begin{block}{}
  	\begin{itemize}
  		\item F41: Profil einstellen
  		\item F42: Lehrmaterialien online stellen und verteilen
  		\item F43: Beurteilungen schreiben
  		\item F44: Schwarzes Brett
  		\newline
  		\item S: Systemanforderungen wurden aus Vorlage übernommen
  	\end{itemize}
  \end{block}

\end{frame}
