\section{SAD}
\subsection{Analysephase}
\begin{frame} %%Eine Folie
  \frametitle{Analyse} %%Folientitel

  Um einen besseren Überblick zu Erhalten wurden die folgenden Fragen gestellt:
% Aufzählung
  \begin{enumerate}
   \item Welche Grundanforderungen sind nötig?
   \item Welche Sprachen werden benötigt?
  \end{enumerate}

\end{frame}
% \addtocounter{enumi}{4}

\begin{frame} %%Eine Folie
  \frametitle{Grundanforderungen - Karte} %%Folientitel

	Für eine Karte sind folgende Funktionen wünschenswert:
%   Auflsitung mit Punkt davor
   \begin{itemize}
    \item Anzeigen des Ziels
    \item Standortabfrage
    \item Routenplanung
   \end{itemize}

\end{frame}

\begin{frame} %%Eine Folie
  \frametitle{Sprachen - Karte} %%Folientitel

%   Geschachtelte Auflsitung mit Punkt davor
   \begin{itemize}
    \item Geolocation
    \begin{itemize}
     \item Google Maps JavaScript API
     \item JavaScript
    \end{itemize}

    \item Routen berechnen
    \begin{itemize}
     \item HTML5 zum Auswählen
     \item JavaScript
    \end{itemize}

    \item Speichern der Standortdaten
    \begin{itemize}
     \item JQuery Cookie-Plugin
     \item JavaScript
    \end{itemize}

    
   \end{itemize}

\end{frame}

\begin{frame} %%Eine Folie
  \frametitle{Grundanforderungen - Video} %%Folientitel

   Um das "`Video"' adäquat nutzen zu können sind folgende Funktionen nötig:
%   Auflsitung mit Punkt davor
   \begin{itemize}
    \item Starten und Anhalten des Videos
    \item Lautstärkeregelung
    \item Spulfunktion
    \item Vollbild
   \end{itemize}

\end{frame}

\begin{frame} %%Eine Folie
  \frametitle{Sprachen - Video} %%Folientitel

%   Geschachtelte Auflsitung mit Punkt davor
   \begin{itemize}
    \item Darstellung
    \begin{itemize}
     \item HTML5 für die verschiedenen Kontrollstrukturen
     \item CSS3
    \end{itemize}

	\item Funktion
	    \begin{itemize}
	     \item JavaScript
	    \end{itemize}
    
   \end{itemize}

\end{frame}

\begin{frame} %%Eine Folie
  \frametitle{Grundanforderungen - Slideshow} %%Folientitel

   Die "`Slideshow"' soll folgende Anforderungen erfüllen:
%   Auflsitung mit Punkt davor
   \begin{itemize}
    \item Fallback Lösung
    \item Dem Thema der Website entsprechend
    
   \end{itemize}

\end{frame}

\begin{frame} %%Eine Folie
  \frametitle{Sprachen - Slideshow} %%Folientitel

%   Geschachtelte Auflsitung mit Punkt davor
   \begin{itemize}
    
    \item Bilder einfügen
        \begin{itemize}
         \item HTML5
        \end{itemize}
    
    \item Fallback-Lösung
    	\begin{itemize}
     	\item Modernizr-Abfrage ob CSS3 möglich
     	\item CSS3
     	\item JavaScript
    	\end{itemize}
   
    
   \end{itemize}

\end{frame}







\subsection{Die Formulierung}
\begin{frame} %%Eine Folie
  \frametitle{Text für die SAD} %%Folientitel

  Aus den Ergebnissen der Analysephase wurde der folgende Text formuliert und in die SAD eingebaut:

% Definitionsblock
  \begin{block}{Karte}
	Mit Hilfe der Google Maps JavaScript API wird eine Karte eingebettet. Außerdem besteht für den Besucher die
	Möglichkeit, sich die Route von seinem jetzigen Standort bis in die Siemensstraße 12 in Regensburg anzeigen zu
	lassen. Um nicht wiederholt die Standortdaten preisgeben zu müssen, werden nach erstmalig erlaubten Zugriff
	diese in einem Cookie gespeichert. Dies geschieht mit Hilfe des jQuerry-Cookie-Plugins.
  \end{block}
  

\end{frame}

\begin{frame} %%Eine Folie
  \frametitle{Text für die SAD} %%Folientitel


% Definitionsblock
 
  
  \begin{block}{Video}
  	Das Video wird mit HTML5 implementiert und ist dadurch ohne Flash abspielbar. Die Kontrollfunktionen
  	werden ebenfalls mit HTML5 zur Verfügung gestellt. Gesteuert wird der HTML5 Videoplayer mit JavaScript.
  \end{block}
    
  \begin{block}{Slideshow}
    Die Bilder der Slideshow werden mit HTML5 zur Verfügung gestellt. Mittels CSS3 werden die Bilder formatiert.
    JQuery sorgt für die Slideanimation. Eine Fallback Lösung, welche in JavaScript implementiert ist, wird durch
    den Modernizr realisiert.
  \end{block}

\end{frame}